% as a Service
\newacronym[
description={
Software as a Service. Modèle du Cloud Computing où l'on paye un abonnement pour utiliser un logiciel qui peut ne pas être présent physiquement sur notre ordinateur
}]{saas}{SaaS}{Software as a Service}

\newacronym[
description={
Platform as a Service. Modèle du Cloud Computing où un environnement d'exécution est mis à disposition du client pour ses propres applications
}]{paas}{PaaS}{Platform as a Service}

\newacronym[
description={
Infrastructure as a Service. Modèle du Cloud Computing où une infrastructure, potentiellement externe, est mis à disposition du client
}]{iaas}{IaaS}{Infrastructure as a Service}

% Services Amazon
\newacronym[
description={
Elastic Compute Cloud. IaaS d'Amazon
}]{ec2}{EC2}{Elastic Compute Cloud}

\newacronym[
description={
Elastic Block Store. Service de stockage bloc d'Amazon pour son IaaS, EC2
}]{ebs}{EBS}{Elastic Block Store}

\newacronym[
description={
Simple Storage Service. Service de stockage objet d'Amazon. Offre une stockage virtuellement illimité dans le cloud
}]{s3}{S3}{Simple Storage Service}

% HP
\newacronym[
description={
Integrated Lights-Out. Système de gestion \emph{out-of-band} d'HP pour ses serveurs
}]{ilo}{iLO}{Integrated Lights-Out}

\newglossaryentry{bladecenter}{
name = {BladeCenter},
description={}
}

% Protocoles
\newacronym[
description={
Secure SHell. Protocole de communication sécurisé permettant le transfert de fichiers entre ordinateurs ou l'administration de serveurs à distance
}]{ssh}{SSH}{Secure SHell}

\newacronym[
description={
Lightweight Directory Access Protocol. Protocole de communication avec les annuaires respectant la norme du même nom. Généralement utilisé pour l'authentification des utilisateurs
}]{ldap}{LDAP}{Lightweight Directory Access Protocol}

\newacronym[
description={
Structured Query Language. Language permettant d'effectuer des requêtes sur des bases de données.
}]{sql}{SQL}{Structured Query Language}

\newacronym[
description={
Pluggable Authentication Modules. Système permettant d'intégrer différent schémas d'authentification à un système UNIX/Linux de façon transparente pour les applications
}]{pam}{PAM}{Pluggable Authentication Modules}

% Virtualisation
\newacronym[
description={
Kernel-based Virtual Machine. Système de virtualisation libre intégré au noyau Linux
}]{kvm}{KVM}{Kernel-based Virtual Machine}

\newacronym[
description={
LinuX Containers
}]{lxc}{LXC}{LinuX Containers}

\newglossaryentry{xen}{
name = {Xen},
description={Logiciel de virtualisation libre}
}

\newglossaryentry{openvz}{
name = {OpenVZ},
description={}
}

\newglossaryentry{hyperv}{
name = {Hyper-V},
description={}
}

% Réseau
\newacronym[
description={
Virtual LAN
}]{vlan}{VLAN}{Virtual LAN}

\newacronym[
description={
Fibre Channel
}]{fc}{FC}{Fibre Channel}

\newacronym[
description={
Fibre Channel
}]{san}{SAN}{Storage area network}

\newglossaryentry{load balancing}{
name = {Load balancing},
description={...}
}

\newglossaryentry{failover}{
name = {Failover},
description={...}
}

% Divers
\newacronym{api}{API}{Application Programming Interface}
\newacronym{crm}{CRM}{Customer Relationship Management}
\newacronym{kvmphys}{KVM}{Keyboard, Video, Mouse}
\newacronym[
description={
}]{raid0}{RAID 0}{}

% Frameworks
\newglossaryentry{cloudstack}{
name = {CloudStack},
description={...}
}

\newglossaryentry{openstack}{
name = {OpenStack},
description={...}
}
\newglossaryentry{proxmox}{
name = {Proxmox},
description={...}
}

% Fondations
\newglossaryentry{fondationopenstack}{
name = {fondation OpenStack},
description={...}
}

\newglossaryentry{fondationapache}{
name = {fondation Apache},
description={...}
}

% Sociétés
\newglossaryentry{cloudcom}{
name = {Cloud.com},
description={...}
}

\newglossaryentry{citrix}{
name = {Citrix},
description={...}
}

% Clouds
\newglossaryentry{cloudcomputing}{
name = {Cloud Computing},
description={...}
}

\newglossaryentry{cloudprive}{
name = {Cloud privé},
description={...}
}

\newglossaryentry{cloudpublic}{
name = {Cloud public},
description={...}
}

% Licences
\newglossaryentry{opensource}{
name = {Open Source},
description={...}
}

\newglossaryentry{licenceapache}{
name = {licence Apache},
description={...}
}



\newglossaryentry{coeur}{
name = {coeur},
plural = {coeurs},
description={En informatique un coeur est une unité de calcul}
}

\newglossaryentry{se}{
name = {système d'exploitation},
plural = {systèmes d'exploitation},
description={...}
}

\newglossaryentry{stockagebloc}{
name = {stockage bloc},
description={...}
}

\newglossaryentry{stockageobjet}{
name = {stockage objet},
description={...}
}


\newglossaryentry{amazon}{
name = {Amazon},
description={...}
}

\newglossaryentry{dropbox}{
name = {Dropbox},
description={Service de stockage et de partage de fichiers en ligne}
}

\newglossaryentry{instagram}{
name = {Instagram},
description={Application de partage de photos pour iOS et Android}
}

\newglossaryentry{netflix}{
name = {Netflix},
description={Service de streaming de films sur Internet}
}

\newglossaryentry{shazam}{
name = {Shazam},
description={Logiciel de reconnaissance musicale}
}

\newglossaryentry{windows}{
name = {Windows},
description={Système d'exploitation de Microsoft}
}

\newglossaryentry{windowsserver}{
name = {Windows Server},
description={Version serveur du système d'exploitation de Microsoft}
}

\newglossaryentry{gappengine}{
name = {Google App Engine},
description={PaaS de Google. Supporte Python, Java, et le Go}
}

\newglossaryentry{msazure}{
name = {Microsoft Azure},
description={IaaS et PaaS de Microsoft}
}

\newglossaryentry{linux}{
name = {Linux},
description={Système d'exploitation libre basé sur le noyau éponyme}
}

\newglossaryentry{solaris}{
name = {Solaris},
description={}
}

\newglossaryentry{bsd}{
name = {*BSD},
description={}
}

\newglossaryentry{parallelsd}{
name = {Parallels Desktop},
description={Logiciel de virtualisation de bureau}
}

\newglossaryentry{vbox}{
name = {VirtualBox},
description={Logiciel de virtualisation de bureau développé par Oracle}
}

\newglossaryentry{php}{
name = {PHP},
description={...}
}

\newglossaryentry{java}{
name = {Java},
description={...}
}

\newglossaryentry{ruby}{
name = {Ruby},
description={Language de programmation orienté objet inspiré du Smalltalk et de Perl. Principalement connu pour la création d'applications web grâce au framework Rails}
}

\newglossaryentry{python}{
name = {Python},
description={Language de programmation orienté objet}
}

\newglossaryentry{scala}{
name = {Scala},
description={...}
}

\newglossaryentry{nodejs}{
name = {Node.js},
description={...}
}

\newglossaryentry{sforce}{
name = {Salesforce},
description={...}
}

\newglossaryentry{sforcecloud}{
name = {Salesforce Sales Cloud},
description={...}
}

\newglossaryentry{heroku}{
name = {Heroku},
description={PaaS de Salesforce. Supporte Ruby, Python, Java, Node.js, Clojure, et Scala}
}

\newglossaryentry{gapps}{
name = {Google Apps for Business},
description={...}
}

\newglossaryentry{tunnelssh}{
name = {tunnel SSH},
description={...}
}

\newglossaryentry{virtualisation}{
name = {virtualisation},
description={...}
}