% as a Service
\newacronym[
description={
Software as a Service. Modèle du Cloud Computing où l'on paye un abonnement pour utiliser un logiciel qui peut ne pas être présent physiquement sur notre ordinateur
}]{saas}{SaaS}{Software as a Service}

\newacronym[
description={
Platform as a Service. Modèle du Cloud Computing où un environnement d'exécution est mis à disposition du client pour ses propres applications
}]{paas}{PaaS}{Platform as a Service}

\newacronym[
description={
Infrastructure as a Service. Modèle du Cloud Computing où une infrastructure, potentiellement externe, est mis à disposition du client
}]{iaas}{IaaS}{Infrastructure as a Service}

% Services Amazon
\newacronym[
description={
Elastic Compute Cloud. IaaS d'Amazon
}]{ec2}{EC2}{Elastic Compute Cloud}

\newacronym[
description={
Elastic Block Store. Service de stockage bloc d'Amazon pour son IaaS, EC2
}]{ebs}{EBS}{Elastic Block Store}

\newacronym[
description={
Simple Storage Service. Service de stockage objet d'Amazon. Offre une stockage virtuellement illimité dans le cloud
}]{s3}{S3}{Simple Storage Service}

% HP
\newacronym[
description={
Integrated Lights-Out. Système de gestion \emph{out-of-band} d'HP pour ses serveurs
}]{ilo}{iLO}{Integrated Lights-Out}

\newglossaryentry{bladecenter}{
name = {BladeCenter},
description={Originellement nom donné à la gamme des serveur blades d'IBM. Désigne par extension tout les systèmes de serveur lames}
}

% Protocoles
\newacronym[
description={
Secure SHell. Protocole de communication sécurisé permettant le transfert de fichiers entre ordinateurs ou l'administration de serveurs à distance
}]{ssh}{SSH}{Secure SHell}

\newacronym[
description={
Lightweight Directory Access Protocol. Protocole de communication avec les annuaires respectant la norme du même nom. Généralement utilisé pour l'authentification des utilisateurs
}]{ldap}{LDAP}{Lightweight Directory Access Protocol}

\newacronym[
description={
Structured Query Language. Language permettant d'effectuer des requêtes sur des bases de données.
}]{sql}{SQL}{Structured Query Language}

\newacronym[
description={
Pluggable Authentication Modules. Système permettant d'intégrer différent schémas d'authentification à un système UNIX/Linux de façon transparente pour les applications
}]{pam}{PAM}{Pluggable Authentication Modules}

\newglossaryentry{tunnelssh}{
name = {tunnel SSH},
description={Un tunnel SSH permet d'encapsuler le trafic IP dans une connexion SSH afin, par exemple, d'accéder à des machines distantes dont l'accès serait bloqué par un pare-feu}
}

\newglossaryentry{syslog}{
name = {Syslog},
description={}
}

\newacronym[
description={
Simple Network Monitoring Protocol.
}]{snmp}{SNMP}{Simple Network Monitoring Protocol}

% Virtualisation
\newglossaryentry{virtualisation}{
name = {virtualisation},
description={En informatique, la virtualisation consiste à faire fonctionner plusieurs systèmes d'exploitations, ou plusieurs applications isolées les une des autres, sur un seul ordinateur}
}

\newglossaryentry{qemu}{
name = {QEMU},
description={Logiciel de virtualisation libre}
}

\newacronym[
description={
Kernel-based Virtual Machine. Système de virtualisation libre basé sur QEMU et utilisant les extensions processeurs de virtualisation. Très performant et supporte tout les systèmes en tant qu'invités
}]{kvm}{KVM}{Kernel-based Virtual Machine}

\newacronym[
description={
LinuX Containers. Système de virtualisation léger intégré au noyau Linux qui permet de créer des environnements exécution isolés. Consomme très peu de ressources et très performant mais ne supporte que les systèmes Linux
}]{lxc}{LXC}{LinuX Containers}

\newglossaryentry{xen}{
name = {Xen},
description={Logiciel de virtualisation libre}
}

\newglossaryentry{openvz}{
name = {OpenVZ},
description={}
}

\newglossaryentry{hyperv}{
name = {Hyper-V},
description={}
}

\newglossaryentry{esxi}{
name = {ESXi},
description={}
}

\newglossaryentry{parallelsd}{
name = {Parallels Desktop},
description={Logiciel de virtualisation de bureau}
}

\newglossaryentry{vbox}{
name = {VirtualBox},
description={Logiciel de virtualisation de bureau développé par Oracle}
}

% Réseau
\newglossaryentry{switch}{
name = {switch},
plural = {switchs},
description={Équipement réseau de couche 2 permettant de connecter d'autres équipements entre eux}
}

\newacronym[
description={
Virtual LAN. Norme permettant d'isoler des réseaux à la couche 2 en utilisant la norme 801.1q afin d'ajouter un identifiant de réseau virtuel au début de la trame Ethernet
}]{vlan}{VLAN}{Virtual LAN}

\newacronym[
description={
Network Address Translation. Mécanisme permettant à des équipements possédant une IP privé de communiquer avec des hôtes situés sur un réseau publique
}]{nat}{NAT}{Network Address Translation}

\newacronym[
description={
Fibre Channel
}]{fc}{FC}{Fibre Channel}

\newglossaryentry{multipath}{
name = {multipath},
description={...}
}

\newacronym[
description={
Storage area network
}]{san}{SAN}{Storage area network}

\newglossaryentry{loadbalancing}{
name = {Load balancing},
description={...}
}

\newglossaryentry{failover}{
name = {Failover},
description={...}
}

% Divers
\newacronym{api}{API}{Application Programming Interface}
\newacronym{crm}{CRM}{Customer Relationship Management}
\newacronym{kvmphys}{KVM}{Keyboard, Video, Mouse}
\newacronym[
description={
}]{raid0}{RAID 0}{}
\newglossaryentry{coeur}{
name = {coeur},
plural = {coeurs},
description={En informatique un coeur est une unité de calcul}
}

% Frameworks
\newglossaryentry{cloudstack}{
name = {CloudStack},
description={...}
}

\newglossaryentry{openstack}{
name = {OpenStack},
description={...}
}
\newglossaryentry{proxmox}{
name = {Proxmox},
description={...}
}

% Fondations
\newglossaryentry{fondationopenstack}{
name = {fondation OpenStack},
description={...}
}

\newglossaryentry{fondationapache}{
name = {fondation Apache},
description={...}
}

% Sociétés
\newglossaryentry{cloudcom}{
name = {Cloud.com},
description={...}
}

\newglossaryentry{citrix}{
name = {Citrix},
description={...}
}

\newglossaryentry{canonical}{
name = {Canonical},
description={...}
}

\newglossaryentry{amazon}{
name = {Amazon},
description={...}
}

\newglossaryentry{hp}{
name = {HP},
description={...}
}

\newglossaryentry{rackspace}{
name = {Rackspace},
description={Hébergeur et fournisseur de solutions de Cloud Computing Américain. Créateur d'OpenStack}
}

\newglossaryentry{ovh}{
name = {OVH},
description={Hébergeur Français}
}

\newglossaryentry{sforce}{
name = {Salesforce},
description={Editeur de logiciel Américain très présent dans le domaine du Cloud Computing}
}

\newglossaryentry{dropbox}{
name = {Dropbox},
description={Service de stockage et de partage de fichiers en ligne}
}

\newglossaryentry{instagram}{
name = {Instagram},
description={Application de partage de photos pour iOS et Android}
}

\newglossaryentry{netflix}{
name = {Netflix},
description={Service de streaming de films sur Internet}
}

\newglossaryentry{shazam}{
name = {Shazam},
description={Logiciel de reconnaissance musicale}
}

% Clouds (voc)
\newglossaryentry{cloudcomputing}{
name = {Cloud Computing},
description={...}
}

\newglossaryentry{cloudprive}{
name = {Cloud privé},
description={...}
}

\newglossaryentry{cloudpublic}{
name = {Cloud public},
description={...}
}

\newglossaryentry{stockagebloc}{
name = {stockage bloc},
description={...}
}

\newglossaryentry{stockageobjet}{
name = {stockage objet},
description={...}
}

\newglossaryentry{glusterfs}{
name = {GlusterFS},
description={...}
}

% Clouds publics
\newglossaryentry{gappengine}{
name = {Google App Engine},
description={PaaS de Google. Supporte Python, Java, et le Go}
}

\newglossaryentry{msazure}{
name = {Microsoft Azure},
description={IaaS et PaaS de Microsoft}
}

\newglossaryentry{sforcecloud}{
name = {Salesforce Sales Cloud},
description={Outil de CRM disponible via Internet}
}

\newglossaryentry{heroku}{
name = {Heroku},
description={PaaS de Salesforce. Supporte Ruby, Python, Java, Node.js, Clojure, et Scala}
}

\newglossaryentry{gapps}{
name = {Google Apps for Business},
description={Suite d'outils pour les entreprises (e-mails, agenda, bureautique) disponible via Internet}
}

% Licences
\newglossaryentry{opensource}{
name = {Open Source},
description={...}
}

\newglossaryentry{licenceapache}{
name = {licence Apache},
description={...}
}

% Systèmes d'exploitation
\newglossaryentry{se}{
name = {système d'exploitation},
plural = {systèmes d'exploitation},
description={...}
}

\newglossaryentry{ubuntu}{
name = {Ubuntu},
description={...}
}

\newglossaryentry{windows}{
name = {Windows},
description={Système d'exploitation de Microsoft}
}

\newglossaryentry{windowsserver}{
name = {Windows Server},
description={Version serveur du système d'exploitation de Microsoft}
}

\newglossaryentry{linux}{
name = {Linux},
description={Système d'exploitation libre basé sur le noyau éponyme}
}

\newglossaryentry{solaris}{
name = {Solaris},
description={}
}

\newglossaryentry{bsd}{
name = {*BSD},
description={}
}

% Languages de prog
\newglossaryentry{php}{
name = {PHP},
description={PHP: Hypertext Preprocessor. Langage de programmation principalement utilisé avec un serveur HTTP pour la création de sites et d'applications web.}
}

\newglossaryentry{java}{
name = {Java},
description={Langage de programmation orienté objet. Utilisé pour sa portabilité car la machine virtuelle Java (JVM) sur laquelle il est basé est disponible sur de nombreuses plateformes. D'après l'indice TIOBE, c'est le langage le plus populaire début 2013}
}

\newglossaryentry{ruby}{
name = {Ruby},
description={Langage de programmation orienté objet inspiré du Smalltalk et de Perl. Principalement connu pour la création d'applications web grâce au framework Rails}
}

\newglossaryentry{python}{
name = {Python},
description={Langage de programmation orienté objet}
}

\newglossaryentry{scala}{
name = {Scala},
description={Langage de programmation inspiré de Java}
}

\newglossaryentry{nodejs}{
name = {Node.js},
description={Logiciel basé sur la machine virtuelle JavaScript V8 de Google permettant d’exécuter du JS côté serveur et ainsi d'écrire des applications web scalables grâce, notamment, aux E/S asynchrones}
}

% Automatisation
\newglossaryentry{serviceorchestration}{
name = {service d'orchestration},
description={Programme qui permet d'automatiser la coordination et l'organisation de systèmes complexe}
}

\newglossaryentry{juju}{
name = {Juju},
description={...}
}

\newglossaryentry{puppet}{
name = {Puppet},
description={...}
}

\newglossaryentry{chef}{
name = {Chef},
description={...}
}

\newglossaryentry{cfengine}{
name = {CfEngine},
description={...}
}

% Logiciels
\newglossaryentry{apache}{
name = {Apache},
description={Serveur HTTP}
}

\newglossaryentry{varnish}{
name = {Varnish},
description={...}
}

\newglossaryentry{mediawiki}{
name = {MediaWiki},
description={CMS permettant la création de Wiki}
}

\newglossaryentry{haproxy}{
name = {HAProxy},
description={...}
}

\newglossaryentry{mysql}{
name = {MySQL},
description={Serveur de base de données}
}

\newglossaryentry{git}{
name = {Git},
description={Système de contrôle de versions distribué. Permet de partager et de synchroniser du code entre plusieurs développeurs}
}